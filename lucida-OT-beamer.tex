\documentclass[svgnames]{beamer}
\useinnertheme[shadow,outline]{chamfered}
\usecolortheme{beaver}
\beamertemplatenavigationsymbolsempty
\usefonttheme{professionalfonts}

\usepackage{amsmath}
\usepackage[bold-style=ISO]{unicode-math}
\defaultfontfeatures{Ligatures=TeX,Scale=.92}
\setmainfont{Lucida Bright OT}
\setsansfont{Lucida Sans OT}
\setmathfont{Lucida Bright Math OT}
\setmathfont[version=bold]{Lucida Bright Math OT Demibold}
\setmonofont[Numbers=SlashedZero]{Lucida Sans Typewriter OT}


\usepackage{listings}
\lstset{%
aboveskip=1ex,
xleftmargin=0.5em,
xrightmargin=3pt,
breaklines=true,
basicstyle={\ttfamily \small},
breakatwhitespace=false,
showspaces=false,
showstringspaces=false
showtabs=false,
keywordstyle=\color{DarkBlue}\bfseries,
stringstyle=\color{DarkRed},
commentstyle=\color{Olive}\itshape,
upquote=true,
}

\definecolor{boxfill}{gray}{0.975}
\definecolor{boxframe}{named}{gray}

\lstnewenvironment{code}
{\lstset{backgroundcolor=\color{boxfill}}}
{}
\lstnewenvironment{python}
{\lstset{language=python,backgroundcolor=\color{boxfill}}}
{}
\lstnewenvironment{bash}
{\lstset{language=bash,backgroundcolor=\color{boxfill}}}
{}
\lstnewenvironment{R}
{\lstset{language=R,backgroundcolor=\color{boxfill}}}
{}



\newcommand{\Mtx}[1]{\ensuremath{\mathbf{#1}}}





\parskip=0.5em


%===========================================================
% Title Info
\title{Scientific Computing for Biologists}
\subtitle{Lecture 10: Mixture Models and Multi-dimensional Scaling} % (optional)

\author{Instructor: Paul M. Magwene}
\date{05 November 2013}


\begin{document}

\begin{frame}
\titlepage
\end{frame}
%===========================================================

%===========================================================
\begin{frame}
  \frametitle{Outline of Lecture}

\begin{itemize}
    \item K-means clustering
    \item Mixture model based clustering
    \item Multi-dimensional scaling (MDS)
\end{itemize}

\end{frame}
%===========================================================
%===========================================================
\begin{frame}
  \frametitle{Gaussian Mixture Models}

A common starting point in mixture modeling is to assume that the components are Gaussian.

If the data are univariate, then the mixture model is given by:

\[
p_{\mathrm{mix}} = \sum_{s=1}^g \pi_s f(\Mtx{x}|\mu_i, \sigma_i^2)
\]

where the $\mu_i$ and $\sigma_i$ are the means and standard deviations of each component distribution and:

\[
f(\Mtx{x}|\mu, \sigma) = \frac{1}{\sqrt{2\pi\sigma^2}}e^{-\frac{(x-\mu)^2}{2\sigma^2}}
\]

\end{frame}

%===========================================================
%===========================================================
\begin{frame}[fragile]
  \frametitle{MDS Example: Road Distances between U.S. Cities}

\begin{center}
\begin{verbatim}
     BOS  CHI   DC  DEN   LA  MIA   NY  SEA   SF
BOS    0  963  429 1949 2979 1504  206 2976 3095
CHI  963    0  671  996 2054 1329  802 2013 2142
DC   429  671    0 1616 2631 1075  233 2684 2799
DEN 1949  996 1616    0 1059 2037 1771 1307 1235
LA  2979 2054 2631 1059    0 2687 2786 1131  379
MIA 1504 1329 1075 2037 2687    0 1308 3273 3053
NY   206  802  233 1771 2786 1308    0 2815 2934
SEA 2976 2013 2684 1307 1131 3273 2815    0  808
SF  3095 2142 2799 1235  379 3053 2934  808    0
\end{verbatim}
\end{center}

\end{frame}
%===========================================================

\begin{frame}[fragile]
\frametitle{Some Code}

Here's some code:

\begin{python}
def myfunc(x):
    print "Hello, Python World"

def yourfunc(x):
    print "Hello, LaTeX World!"

x = 10
myfunc(x)

\end{python}


\end{frame}

%========================================

%========================================
\begin{frame}[fragile]
  \frametitle{Last Slide}

  This is the \emph{last} slide of the document.

It was created using:
  \begin{itemize}
  \item YASnippet
  \item AucTeX 
  \item XeLaTeX
  \end{itemize}

\end{frame}
%========================================




%========================================
\end{document}