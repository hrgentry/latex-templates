%-----------------------------------------------

\newcommand{\normdistn}{\ensuremath{%
f(x) = 
\frac{1}{\sqrt{2\pi\sigma^2}} e^{-\frac{(x-\mu)^2}{2\sigma^2}}
}}

\newcommand{\varx}{\ensuremath{\frac{1}{n}\sum_{i=1}^{n}(x_i-\mu)}}


%-----------------------------------------------
\title{A \LaTeX\ Test Document}
\author{Paul M. Magwene}

\begin{document}
\maketitle
    
\section{`The Crisis' by Thomas Paine}

These are the times that try men's souls. The summer soldier and the sunshine patriot will, in this crisis, shrink from the service of their country; but he that stands by it now, deserves the love and thanks of man and woman. Tyranny, like hell, is not easily conquered; yet we have this consolation with us, that the harder the conflict, the more glorious the triumph. What we obtain too cheap, we esteem too lightly: it is dearness only that gives every thing its value....


\section{Math}

Here's a simple equation: $\cos^2 \varphi + \sin^2 \varphi = -e^{i\pi}$ and here is a an equation with a sum: \varx. We follow these up with a `display equation' as shown below:
\[
\normdistn
\]

Let's do some typesetting of number, for example: 11 and 17. Now, here's how those look in math mode: $11$ and $17$. Are they any different? If so, which looks better?

\section{Programming}

And here is some verbatim:
\begin{verbatim}
    import math
    x = math.cos(2*math.pi)
    l = 1.1 * tan(x)
    print "Can you tell the difference between O and 0?"
    print "How about 1 and l?"
\end{verbatim}  
    
\end{document}