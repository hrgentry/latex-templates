%!TEX TS-program = xelatex
\documentclass[12pt,letterpaper]{scrartcl}
\usepackage[margin=1.25in]{geometry}

\usepackage{mathspec} % includes fontspec
\usepackage{microtype}
\usepackage{unicode-math}

\defaultfontfeatures{Scale=MatchLowercase,Numbers={Lining,Proportional}}
\setmainfont[Renderer=Basic,Ligatures=TeX]{Adobe Garamond Pro} 

% A number of sans serif fonts look decent w/Libertine
%\setsansfont{Linux Biolinum O}
%\setsansfont{DejaVu Sans Condensed}
%\setsansfont{Gill Sans Light} 
\setsansfont[Scale=0.86]{Cronos Pro}
%\setsansfont{Optima}
\setmonofont[Scale=0.8]{News Gothic Std}
\setmathfont{Asana-Math.otf}
%\setmathsfont(Digits,Latin,Greek){Linux Libertine O}

\usepackage{xcolor}
\definecolor{spot}{rgb}{0.6,0,0}
\definecolor{boxframe}{rgb}{0.6,0,0}
\definecolor{boxfill}{rgb}{1,.95,.95}

\usepackage{listings}
\lstset{%
     aboveskip=1.5em,
     xleftmargin=3pt,
     xrightmargin=3pt,
     frame=single,
     backgroundcolor=\color{boxfill},
     rulecolor=\color{boxframe},
     framesep=2.5pt,
     framerule=0.5pt,
     language=Python,
     breaklines=true,
    basicstyle={\ttfamily \small},                         
breakatwhitespace=false,                
showspaces=false,                       
showstringspaces=false
showtabs=false}




\newcommand{\laplace}{$$\int_0^\infty \mathup e^{-st}f(t)$$}



\begin{document}

\section{Thomas Paine}
    
    These are the times that try men's souls. The summer soldier, and the sunshine patriot, will in this crisis shrink from the service of his country. But he that stands it now deserves the love and thanks of man and woman...

% \setsansfont{Optima}
% \section{Thomas Paine in Optima}
% 
% \setsansfont{Lucida Sans}
% \section{Thomas Paine in Lucida Sans}

\section{Math}

Here's a simple equation: $\cos^2 \varphi + \sin^2 \varphi = -e^{i\pi}$. And a display equation is below:
    \laplace
And here is some verbatim:
\begin{lstlisting}
    import math
    x = math.cos(2*math.pi)
    l = 1.1 * tan(x)
    print "Can you tell the difference between O and 0?"
    print "How about 1 and l?"
\end{lstlisting}  

Is 11 different than 17? Do en dash -- and em dash --- work?

    
\end{document}